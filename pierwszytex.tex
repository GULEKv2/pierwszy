\documentclass[12pt,a4paper]{article}
\usepackage[MeX]{polski} 
\usepackage[utf8]{inputenc}  

\title{Nie za krótkie wprowadzenie do \LaTeX}
\author{Łukasz Żurek}
\date{28.01.2022}
\begin{document}
\maketitle

\begin{abstract}
    To jest streszczenie, pamiętaj.
	$ 3 - 4 = -1$
\end{abstract}
\newpage
\textbf {jest jak jest.}
	\texttt {co możemy zrobić}
\tableofcontents
odnoszę się teraz do artykułu i jest dobry 
\newpage
\section{zabawa i takie tam }
	\texttt {trzeba z tym życ}
\begin{tabular}{|l|cr|}
\hline tebela mówi nam o tym że jestem w trakcie uczenia się o programie Latex \\ \hline
\hline pamiętaj \\ \hline
\hline Latex \\ \hline
\end{tabular}
	\subsection{zabawa zabawy i takie tam takiego tam}
		\textbf {nic nie poradzimy}
\newpage
\section{matematyka}
	\textrm {$ 2 + 3 = 223 $}
	\textbf {slynne równanie:} \\
	$E = mc^2$
	\texttt {polecam}
\newpage


\begin{thebibliography}{9}

\bibitem{lamport94}
  Miot Mateusz,
  \emph{\LaTeX: Prezentacja Latex}.
  Pomorskie, Gdańsk,
  1 edycja,
  6 grudnia 2006.

\end{thebibliography}

\end{document}