\documentclass{beamer}
\usetheme{Frankfurt}
\usecolortheme{beetle}
\usepackage[polish]{babel}
\usepackage[utf8]{inputenc}
\usepackage[T1]{fontenc}
 \title{Jest Tytuł}
 \author{Łukasz Żurek}
\date{28.01.2022}

 \begin{document}
 \begin{frame}
\titlepage
\end{frame}
\begin{frame}{Tytuł}
Piękny i wspaniały slajd
\begin{tabular}{ |l |l |l |l |l|} 
\hline
	

  kto wie & nie & moze & zabawa & tak  \\ \hline
  kto wie & nie & moze & zabawa & tak   \\ \hline
  kto wie & nie & moze & zabawa  & tak  \\ \hline
  kto wie &nie & moze & zabawa & tak  \\ \hline
    kto wie & nie & moze & zabawa & tak   \\ \hline
\end{tabular}

\end{frame}
\begin{frame}{Tytuł}
Hipopotam

Zachwycony jej powabem
Hipopotam błagał żabę:

"Zostań żoną moją, co tam,
Jestem wprawdzie hipopotam,
Kilogramów ważę z tysiąc,
Ale za to mógłbym przysiąc,
Że wzór męża znajdziesz we mnie
I że ze mną żyć przyjemnie.
Czuję w sobie wielki zapał,
Będę ci motylki łapał
I na grzbiecie, jak w karecie,
Będę woził cię po świecie,
A gdy jazda już cię znuży,
Wrócisz znowu do kałuży.
Krótko mówiąc - twoją wolę
Zawsze chętnie zadowolę,
Każdy rozkaz spełnię ściśle.
Co ty na to?"

\begin{figure}
   
    
   
    \label{fig:Stąd się właśnie pewność bierze,
Że nie jestem ptak ni zwierzę,}
\end{figure}

Hipopotam

Zachwycony jej powabem
Hipopotam błagał żabę:

"Zostań żoną moją, co tam,
Jestem wprawdzie hipopotam,
Kilogramów ważę z tysiąc,
Ale za to mógłbym przysiąc,
Że wzór męża znajdziesz we mnie
I że ze mną żyć przyjemnie.
\end{frame}

\begin{frame}{Tytuł}



Będę ci motylki łapał
I na grzbiecie, jak w karecie,
Będę woził cię po świecie,
A gdy jazda już cię znuży,
Wrócisz znowu do kałuży.
Krótko mówiąc - twoją wolę
Zawsze chętnie zadowolę,
Każdy rozkaz spełnię ściśle.
Co ty na to?"

\begin{figure}
   
   
    \label{fig:Pies rozumie, bo ja wiem,
Jak rozmawiać trzeba z psem.
.}
\end{figure}
 Jak rozmawiać trzeba z psem?

Wy nie wiecie, a ja wiem,
Jak rozmawiać trzeba z psem,

Bo poznałem język psi,
Gdy mieszkałem w pewnej wsi.
\end{frame}
\begin{frame}{Tytuł}
Jak rozmawiać trzeba z psem?

Wy nie wiecie, a ja wiem,
Jak rozmawiać trzeba z psem,

Bo poznałem język psi,
Gdy mieszkałem w pewnej wsi.

A więc wołam: - Do mnie, psie!
I już pies odzywa się.

Potem wołam: - Hop-sa-sa!
I już mam przy sobie psa.

A gdy powiem: - Cicho leż!
Leżę ja i pies mój też.

Kiedy dłoń wyciągam doń,
Grzecznie liże moją dłoń.

I zabawnie szczerzy kły,
Choć nie bywa nigdy zły.


 \begin{enumerate}
  \item tak
  \pause
  \item nie wiem 
  \pause
  \item przedmiot
  \pause
\end{enumerate} in excelsis deus 
\end{frame}
\begin{frame}{Tytuł}
LOREM
\begin{itemize}
  \item Gdy pisałem wierszyk ten,
Pies u nóg mych zapadł w sen,

Potem wstał, wyprężył grzbiet,
Żebym z nim na spacer szedł.
  \pause
  \item A gdy powiem: - Cicho leż!
Leżę ja i pies mój też.

  \pause
  \item Szliśmy razem - ja i on,
Pies postraszył stado wron,

\end{itemize}
\begin{figure}
   

   
    \label{fig:A gdy powiem: - Cicho leż!
Leżę ja i pies mój też.}
\end{figure}
\end{frame}
\begin{frame}{Tytuł}
Kiedy dłoń wyciągam doń,
Grzecznie liże moją dłoń.

I zabawnie szczerzy kły,
Choć nie bywa nigdy zły.

Gdy psu kość dam - pies ją ssie,
Bo to są zwyczaje psie.
\begin{figure}
    \centering

    \caption{Tylko człowiek, starszy pan,
Który zwie się - Brzechwa Jan.}
    \label{fig:Gdy psu kość dam - pies ją ssie,
Bo to są zwyczaje psie.}
\end{figure}
\end{frame}
\begin{frame}{Tytuł}
Wy nie wiecie, a ja wiem,
Jak rozmawiać trzeba z psem,

Bo poznałem język psi,
Gdy mieszkałem w pewnej wsi.
\begin{figure}
    \centering

  
    \label{fig:Co jest lepsze? Ręce cztery?
Cztery nogi? Będę szczery}
\end{figure}
Potem wstał, wyprężył grzbiet,
Żebym z nim na spacer szedł.

Szliśmy razem - ja i on,
Pies postraszył stado wron,

Potem biegł zwyczajem psim,
A ja biegłem razem z nim.
\end{frame}

\begin{frame}{Tytuł}
\begin{thebibliography}{szacunek}



\bibitem{one} Jan Brzechwa,
\textit{ wierszykidladzieci.pl}.
PWN, Żmerynka, 1950.
\bibitem{two}N.Nowak,
\textit{Świecie}.
ciekawe
\bibitem{three}
K.Kowalski, \textit{Gdańsk}, 
1231.
\bibitem{four}
J.Jankowski,\textit{Warszawa}

\end{thebibliography}{}

 \end{frame}

 
 \end{document}
